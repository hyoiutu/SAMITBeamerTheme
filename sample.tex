\documentclass[dvipdfmx]{beamer}
\usetheme{SAMIT}

\title{Beamerのテスト}
\subtitle{Beamerのデザインを自作する}
\author{FujiwaLaTeX}
\institute{室蘭工業大学大学院工学研究科 情報電子工学系専攻}

\begin{document}
\maketitle
\section{アイゼンシュタイン級数}
\begin{frame}{ゼータ関数(frametitle)}
  \begin{block}{ゼータ関数(block環境)}\alert{
    \begin{equation*}
      \zeta(s) = \sum_{n=0}^{\infty} \frac{1}{n^s} = 1 + \frac{1}{2^s} + \frac{1}{3^s} + \cdots
    \end{equation*}}
  \end{block}
  \begin{exampleblock}{$n = 2$(exampleblock環境)}
    \begin{equation*}
      \zeta(2) = \sum_{n=0}^{\infty} \frac{1}{n^2} = \frac{\pi^2}{6}
    \end{equation*}
  \end{exampleblock}
  \begin{alertblock}{正の偶数$2n$の時(alertblock環境)}
    \begin{equation*}
      \zeta(2n) = (-1)^{n+1} \frac{B_{2n}(2\pi)^{2n}}{2(2n)!}
    \end{equation*}
  \end{alertblock}
\end{frame}

\begin{frame}{アイゼンシュタイン級数}
  \begin{equation*}
    G_{2k}(\tau) = \sum_{(m,n) \in \mathbb{Z}^2 \backslash (0,0)}\frac{1}{(m+n\tau)^{2k}}
  \end{equation*}
\end{frame}
\end{document}
